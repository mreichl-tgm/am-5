\documentclass{school}

\title{Stochastik}
\subject{Angewandte Mathematik}
\author{Markus Reichl}

\begin{document}
\maketitle
\thispagestyle{fancy}	% Makes the first page fancy too

\tableofcontents

\newpage
\section{Grundlagen}
Der Zufall unterliegt Regeln, die erst bei einer großen Anzahl von Versuchen sichtbar werden.\\
Ein Zufallsexperiment muss $n$-Mal durchgeführt werden um zu erkennen, dass das Ereignis $A$ $k$-Mal auftritt.

Nach einer ausreichenden Zahl von Versuchen kann man als Schätzwert für die Wahrscheinlichkeit $P(A)$, die Anzahl der Ereignisse $k$ relativ zur Anzahl der Versuche $n$ nehmen.
$$P(A) = \frac{k}{n}$$
{\footnotesize $$0 \leqslant P(A) \leqslant 1$$}
\begin{vardefs}
\addvardef{$P(A)$}{Wahrscheinlichkeit}
\addvardef{$A$}{Ereignis}
\addvardef{$k$}{Anzahl der Ereignisse}
\addvardef{$n$}{Anzahl der Versuche}
\end{vardefs}

\section{Laplace Experiment}
Gilt für ein Zufallsexperiment, dass
\begin{outline}
\1 endlich viele Ereignisse existieren und
\1 jedes Ereignis gleich wahrscheinlich ist,
\end{outline}
kann man ein Laplace Experiment durchführen. Man unterscheidet dabei die Zahl der günstigen $g$ und der möglichen $m$ Fälle für ein Ereignis $A$. Die Wahrscheinlichkeit $P(A)$ für ein Ereignis $A$ ist als Quotient der günstigen $g$ und der möglichen $m$ Fälle definiert.
$$P(A) = \frac{g}{m}$$

\section{Zusammengesetzte Ereignisse}
2 Ereignisse $A, B$ sollen zu einem Ereignis $C$ zusammengesetzt werden.
\begin{center}
\begin{tabularx}{0.5\textwidth}{c c c}
	$A$ UND $B$ & bzw. & $A$ ODER $B$\\
	{\small ($A$ und $B$ zugleich)} & & {\small (mindestens $A$ oder $B$)}
\end{tabularx}
\end{center}

\paragraph{Gegenwahrscheinlichkeit} $P(\bar{A}) = 1 - P(A)$ \quad $\bar{A}$ \quad \dots \quad "Non-A"

\paragraph{Unvereinbar} 2 Ereignisse sind unvereinbar wenn sie nicht gemeinsam Auftreten können ($P(A_{\text{UND}}B) = 0$).

\paragraph{Additionssatz \small{(ODER-Regel)}}
$$P(A_{\text{ODER}}B) = P(A) + P(B) - P(A_{\text{UND}}B) \quad \text{wenn A und B beliebig}$$
$$P(A_{\text{ODER}}B) = P(A) + P(B) \quad \text{wenn A und B unvereinbar}$$

\paragraph{Multiplikationssatz \small{(UND-Regel)}}
$$P(A_{\text{UND}}B) = P(A) * P(B_{\text{ODER}}A) \quad \text{wenn A und B beliebig}$$
$$P(A_{\text{UND}}B) = P(A) * P(B) \quad \text{wenn A und B unvereinbar}$$

\section{Bernoulli Versuch \normalsize{(Pfade)}}
\paragraph{1. Pfadregel \small{UND-Regel}} Die Wahrscheinlichkeit eines Ereignisses ist gleich dem Produkt der Wahrscheinlichkeiten entlang seines Pfades.
\paragraph{2. Pfadregel \small{ODER-Regel}} Die Wahrscheinlichkeit eines Ereignisses ist gleich der Summe der Wahrscheinlichkeiten seiner Pfade.

\section{Abzähltechniken}
\subsection{Permutation}
Es seien 4 Personen gegeben und es wird die Anzahl verschiedener Reihenfolgen gesucht. Jede Person kann nur an einer Position stehen, es handelt sich also um ein \textit{ziehen ohne zurücklegen}.
\\\\
Nach jeder Auswahl einer Position stehen eine Person und ein Platz weniger zur Verfügung.

{\small \begin{tabular}{l l l l l}
1. & 2. & 3. & 4. & Position\\
4 & 3 & 2 & 1 & Möglichkeiten
\end{tabular}}
\\\\
Es bestehen also $4 * 3 * 2 * 1$ Möglichkeiten (Permutationen) oder $4$ Fakultät ($4!$).\\
Die Anzahl der Permutationen von $n$ Elementen entspricht also $n! = \sum_{i=0}^{n - 1} n-i$.

\subsection{Kombination}
Aus $n=7$ Personen sollen $k=3$ Personen ausgewählt werden. Die Reihenfolge ist dabei belanglos, es handelt sich also um ein \textit{ziehen mit zurücklegen}.

{\small \begin{tabular}{l l}
$7 * 6 * 5$ & sind wählbar\\
$3 * 2 * 1$ & sind gleich für jede Auswahl
\end{tabular}}
\\\\
Es bestehen also $\frac{7 * 6 * 5}{3 * 2 * 1} = \frac{210}{6} = 35$ Möglichkeiten.
Allgemein ist die Kombination als $\frac{n * (n-1) \text{\dots} (n - k + 1)}{1 \text{\dots} k}$ definiert, was dem Binomialkoeffizienten entspricht.

\paragraph{Binomialkoeffizient}
$$\binom{n}{k} = \frac{n!}{(n-k)! * k!}$$
In wxMaxima kann der Binomialkoeffizient mittels \verb|binomial(n, k)| berechnet werden.

\subsection{Variation \normalsize{(Geordnete Auswahl)}}
Aus $n=7$ Personen sollen $k=3$ ausgewählt werden. Die Reihenfolge der Kombination ist dabei zu beachten, es handelt sich also um ein \textit{ziehen ohne zurücklegen}.

{\small \begin{tabular}{l l l l}
1. & 2. & 3. & Position\\
$7$ & $6$ & $5$ & $\rightarrow 7 * 6 * 5$ Möglichkeiten
\end{tabular}}
\\\\
Die Variation kann auch anhand der Kombination berechnet werden.
$$\binom{7}{3} * 3! = 7 * 6 * 5 = 210 \text{~Variationen}$$

\newpage
\section{Verteilung}
\subsection{Hypergeometrische Verteilung}


% Basic Figure
% \begin{figure}[h]
%	 \centering
% 	 \includegraphics[height=4cm]{image.jpg}
% 	 \caption{Caption}
% \end{figure}

% Basic bibiography
% \begin{thebibliography}{9}
% \bibitem{faz} faz.net, Vergewaltigung live auf Facebook gezeigt \\ http://www.faz.net/aktuell/gesellschaft/kriminalitaet/vergewaltigung-live-auf-facebook-gezeigt-14936872.html
% \end{thebibliography}

% List of figures
% \listoffigures
\end{document}